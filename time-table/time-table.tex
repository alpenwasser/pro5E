\documentclass[10pt,landscape]{article}


% ---------------------------------------------------------------------------- %
% Packages
% ---------------------------------------------------------------------------- %
\usepackage[T1]{fontenc}
\usepackage[utf8]{inputenc}
%\usepackage[ngerman]{babel}
\usepackage[english]{babel}
\usepackage[a3paper,height=250mm,width=350mm]{geometry}
\usepackage[affil-it]{authblk}
\usepackage{sectsty}
\usepackage{lmodern}
\usepackage{pgfgantt}
\fontfamily{lmr}\selectfont
\subsectionfont{\fontsize{11}{12}\selectfont}
%\usepackage{xcolor-solarized}
%\pagecolor{solarized-base02}
%\color{solarized-base2}
%\pagecolor{solarized-base2}
%\color{solarized-base02}
%\renewcommand{\familydefault}{\sfdefault}
%\usepackage{siunitx}


% ---------------------------------------------------------------------------- %
% Macros
% ---------------------------------------------------------------------------- %
\def\code#1{\texttt{#1}}

% Required for counting the correct KW in the gantt chart
\newcounter{calendarWeekNum}
\stepcounter{calendarWeekNum}
\newcommand{\calendarWeek}{\thecalendarWeekNum
    \stepcounter{calendarWeekNum}
    \ifnum\thecalendarWeekNum=53
         \setcounter{calendarWeekNum}{1}
    \else\fi
}


% ---------------------------------------------------------------------------- %
% Title Setup
% ---------------------------------------------------------------------------- %
\title{Project 5 -- Sensor-Microchip}
\author{Raphael Frey, Alexander Murray}
\affil{Institute for Mikroelektronics\\FHNW -- Hochschule f\"ur Technik\\ Studiengang EIT}

% **************************************************************************** %
% Content
% **************************************************************************** %
\begin{document}

\maketitle

% Our project begins in week 41
\setcounter{calendarWeekNum}{38}
\ganttset{%
    calendar week text={\calendarWeek{}}%
}

\begin{figure}[h!bt]
    \begin{center}
        \begin{ganttchart}[
            vgrid={*{6}{draw=none}, dotted},
            x unit=.20cm,
            y unit title=.6cm,
            y unit chart=.6cm,
            time slot format=isodate,
            time slot format/start date=2016-10-10
        ]{2016-09-19}{2017-01-30}
            \ganttset{bar height=.6}
            \gantttitlecalendar{year, month=name, week} \\
            \ganttbar[bar/.append style={fill=blue}]{Acquire fundamental knowledge}{2016-09-23}{2016-10-21} \\
            \ganttbar[bar/.append style={fill=blue}]{Skeleton of Report}{2016-10-10}{2016-10-12} \\
            \ganttbar[bar/.append style={fill=red}]{Write Draft Report}{2016-10-13}{2016-12-02} \\
            \ganttbar[bar/.append style={fill=red}]{Write Report}{2016-12-03}{2017-01-13} \\
            \ganttbar[bar/.append style={fill=blue}]{Prepare Preamp measurement of the Burgherr Chip}{2016-10-17}{2016-10-23} \\
            \ganttbar[bar/.append style={fill=blue}]{Prepare ADC measurement of the Burgherr Chip}{2016-10-17}{2016-10-23} \\
            \ganttbar[bar/.append style={fill=blue}]{Measure Preamp in Burgherr Chip}{2016-10-24}{2016-12-23} \\
            \ganttbar[bar/.append style={fill=blue}]{Measure ADC in Burgherr Chip}{2016-10-24}{2016-12-23} \\
            \ganttbar[bar/.append style={fill=blue}]{Learn how to use simulations}{2016-10-31}{2016-11-27} \\
            \ganttbar[bar/.append style={fill=blue}]{Evaluate results of the measurements}{2016-11-14}{2017-01-08} \\
            \ganttbar[bar/.append style={fill=blue}]{Prepare measurements for P6}{2017-01-09}{2017-01-22} \\
            \ganttbar[bar/.append style={fill=blue}]{Create website}{2016-11-28}{2016-12-11} \\
            \ganttbar[bar/.append style={fill=blue}]{Fill website with content}{2016-12-31}{2017-01-13} \\
            \ganttbar[bar/.append style={fill=blue}]{Prepare presentation}{2017-01-23}{2017-01-30} \\
            \ganttmilestone{First measurements}{2016-10-21} \\
            \ganttmilestone{Delivery of Draft}{2016-12-03} \\
            \ganttmilestone{Delivery of Documentation}{2017-01-13} \\
            \ganttmilestone{Presentation}{2017-01-30} \\
        \end{ganttchart}
    \end{center}
    \label{fig:time-plan}
    \caption{Time Plan}
\end{figure}

\begin{itemize}
    \item
        Basic truss of report.
        1 day
    \item
        Acquire fundamental knowledge.
        The entire project
    \item   
        Write a "theory" block in the report. Purpose is to help acquire the fundamental knowledge.
    \item
        Preparation of measurement of the Burgherr chip. This includes: Figure out what exactly we want to measure, and how we measure.
        -> Look at existing ADCs, see what typical measurements are performed, see what is typically used to characterise an ADC.
    \item
        Same as above, s/ADC/Preamp/
    \item   
        Learn how to use the simulations, so we can compare the models to the actual measured results.
    \item
        Performing the measurements of the Burgherr chip.
    \item
        Evaluate results of the measurements and draw conclusions of the Burgherr chip.
    \item
        Figure out basic structure of the report (Fachbericht).
    \item
        Prepare measurements for P6, for the second chip.
    \item
        Create website.
    \item
        Final report.
    \item
        Create/prepare presentation.
\end{itemize}

\end{document}

