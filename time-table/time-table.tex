\documentclass[10pt,landscape]{article}


% ---------------------------------------------------------------------------- %
% Packages
% ---------------------------------------------------------------------------- %
\usepackage[T1]{fontenc}
\usepackage[utf8]{inputenc}
%\usepackage[ngerman]{babel}
\usepackage[english]{babel}
\usepackage[a3paper,height=250mm,width=350mm]{geometry}
\usepackage[affil-it]{authblk}
\usepackage{sectsty}
%\usepackage{lmodern}
\usepackage{kpfonts}
\usepackage{pgfgantt}
\usepackage{xcolor-solarized}
\fontfamily{lmr}\selectfont
\subsectionfont{\fontsize{11}{12}\selectfont}
%\usepackage{xcolor-solarized}
%\pagecolor{solarized-base02}
%\color{solarized-base2}
%\pagecolor{solarized-base2}
%\color{solarized-base02}
%\renewcommand{\familydefault}{\sfdefault}
%\usepackage{siunitx}


% ---------------------------------------------------------------------------- %
% Macros
% ---------------------------------------------------------------------------- %
\def\code#1{\texttt{#1}}

% Requisolarized-cyan for counting the correct KW in the gantt chart
\newcounter{calendarWeekNum}
\stepcounter{calendarWeekNum}
\newcommand{\calendarWeek}{\thecalendarWeekNum
    \stepcounter{calendarWeekNum}
    \ifnum\thecalendarWeekNum=53
         \setcounter{calendarWeekNum}{1}
    \else\fi
}


% ---------------------------------------------------------------------------- %
% Title Setup
% ---------------------------------------------------------------------------- %
\title{Project 5 -- Timetable}
\author{Raphael Frey, Alexander Murray}
\affil{Institute for Mikroelektronics\\FHNW -- Hochschule f\"ur Technik\\ Studiengang EIT}

% **************************************************************************** %
% Content
% **************************************************************************** %
\begin{document}

%\maketitle

\begin{center}
    \Huge{\textbf{Timetable}}\\
    \vspace{5mm}
    \large{Project 5 -- Raphael Frey, Alexander Murray}
\end{center}
\vspace{20mm}

% Our project begins in week 38
\setcounter{calendarWeekNum}{38}
\ganttset{%
    calendar week text={\calendarWeek{}}%
}

\begin{figure}[h!bt]
    \begin{center}
        \begin{ganttchart}[
            vgrid={*{6}{draw=none}, dotted},
            x unit=2mm,
            y unit title=10mm,
            y unit chart=8mm,
            time slot format=isodate,
            time slot format/start date=2016-10-10
        ]{2016-09-19}{2017-01-30}
            \ganttset{bar height=.6}
            \gantttitlecalendar{year, month=name, week} \\
            \ganttbar[bar/.append style={fill=solarized-blue}]{Acquire fundamental knowledge}{2016-09-23}{2016-10-21} \\
            \ganttbar[bar/.append style={fill=solarized-magenta}]{Prepare Preamp measurement of the Burgherr Chip}{2016-10-17}{2016-10-23} \\
            \ganttbar[bar/.append style={fill=solarized-magenta}]{Prepare ADC measurement of the Burgherr Chip}{2016-10-17}{2016-10-23} \\
            \ganttlinkedmilestone{First measurements}{2016-10-23} \\
            \ganttlinkedbar[bar/.append style={fill=solarized-magenta}]{Measure Preamp in Burgherr Chip}{2016-10-24}{2016-12-23} \\
            \ganttbar[bar/.append style={fill=solarized-magenta}]{Measure ADC in Burgherr Chip}{2016-10-24}{2016-12-23} \\
            \ganttbar[bar/.append style={fill=solarized-blue}]{Learn how to use simulations}{2016-10-31}{2016-11-27} \\
            \ganttbar[bar/.append style={fill=solarized-blue}]{Evaluate results of the measurements}{2016-11-14}{2017-01-08} \\
            \ganttbar[bar/.append style={fill=solarized-blue}]{Prepare measurements for P6}{2017-01-09}{2017-01-22} \\
            \ganttbar[bar/.append style={fill=solarized-blue}]{Create website}{2016-11-28}{2016-12-11} \\
            \ganttlinkedbar[bar/.append style={fill=solarized-blue}]{Fill website with content}{2016-12-31}{2017-01-13} \\
            \ganttbar[bar/.append style={fill=solarized-cyan}]{Skeleton of Report}{2016-10-10}{2016-10-12} \\
            \ganttlinkedbar[bar/.append style={fill=solarized-cyan}]{Write Draft Report}{2016-10-13}{2016-12-03} \\
            \ganttlinkedmilestone{Delivery of Draft}{2016-12-03} \\
            \ganttlinkedbar[bar/.append style={fill=solarized-cyan}]{Write Report}{2016-12-04}{2017-01-13} \\
            \ganttlinkedmilestone{Delivery of Documentation}{2017-01-13} \\
            \ganttbar[bar/.append style={fill=solarized-blue}]{Prepare presentation}{2017-01-23}{2017-01-30} \\
            \ganttlinkedmilestone{Presentation}{2017-01-30} \\

        \end{ganttchart}
    \end{center}
    %\label{fig:time-plan}
    %\caption{Timetable}
\end{figure}

%\begin{itemize}
%    \item
%        Basic truss of report.
%        1 day
%    \item
%        Acquire fundamental knowledge.
%        The entire project
%    \item
%        Write a "theory" block in the report. Purpose is to help acquire the fundamental knowledge.
%    \item
%        Preparation of measurement of the Burgherr chip. This includes: Figure out what exactly we want to measure, and how we measure.
%        -> Look at existing ADCs, see what typical measurements are performed, see what is typically used to characterise an ADC.
%    \item
%        Same as above, s/ADC/Preamp/
%    \item
%        Learn how to use the simulations, so we can compare the models to the actual measusolarized-cyan results.
%    \item
%        Performing the measurements of the Burgherr chip.
%    \item
%        Evaluate results of the measurements and draw conclusions of the Burgherr chip.
%    \item
%        Figure out basic structure of the report (Fachbericht).
%    \item
%        Prepare measurements for P6, for the second chip.
%    \item
%        Create website.
%    \item
%        Final report.
%    \item
%        Create/prepare presentation.
%\end{itemize}

\end{document}

