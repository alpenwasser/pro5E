\documentclass[a4paper,10pt]{article}


% ---------------------------------------------------------------------------- %
% Packages
% ---------------------------------------------------------------------------- %
\usepackage[T1]{fontenc}
\usepackage[utf8]{inputenc}
%\usepackage[ngerman]{babel}
\usepackage[english]{babel}
\usepackage[a4paper,height=250mm,width=150mm]{geometry}
\usepackage[affil-it]{authblk}
\usepackage{sectsty}
\usepackage{lmodern}
\usepackage{enumitem}
\usepackage{lipsum}
\fontfamily{lmr}\selectfont
\subsectionfont{\fontsize{11}{12}\selectfont}
%\usepackage{xcolor-solarized}
%\pagecolor{solarized-base02}
%\color{solarized-base2}
%\pagecolor{solarized-base2}
%\color{solarized-base02}
%\renewcommand{\familydefault}{\sfdefault}
%\usepackage{siunitx}


% ---------------------------------------------------------------------------- %
% Macros
% ---------------------------------------------------------------------------- %
\def\code#1{\texttt{#1}}


% ---------------------------------------------------------------------------- %
% Title Setup
% ---------------------------------------------------------------------------- %
\title{Project 5 -- Sensor-Microchip}
\author{Raphael Frey, Alexander Murray}
\affil{Institute of Microelektronics\\FHNW School for Engineering}



% **************************************************************************** %
% Content
% **************************************************************************** %
\begin{document}

% ---------------------------------------------------------------------------- %
\maketitle
% ---------------------------------------------------------------------------- %

% ---------------------------------------------------------------------------- %
% \section{Ziele}
% \label{sec:ziele}
% ---------------------------------------------------------------------------- %

%\vspace{1em}

\begin{description}[align=left]
    \item[Current State of Research:]
        In his  Master Thesis  \emph{Sensor Chip}  from 2015,  Tobias Burgherr
        developed  a  sensor   integrated  circuit  as  a   base  for  further
        projects  to use. This  IC  consists of  a programmable  pre-amplifier
        and   a  sigma-delta   analog-to-digital  converter   with  adjustable
        bandwidth. Typically, sigma-delta ADCs use an internal 1-bit quantizer
        which gets boosted  to the desired resolution of  bits via delta-sigma
        modulation. The goal  of his work  was to replace the  1-bit quantizer
        with  a  successive  approximation   register  (SAR)  ADC  capable  of
        converting 3  bits at a time  instead of 1.  This  method was referred
        to  as  a  ``Multi-bit  quantizer''. A  complementary  3-bit  DAC  was
        also  developed to  be used  internally in  the feedback  loop of  the
        sigma-delta  modulator. Based  on his  work,  two  teams modified  the
        existing chip in 2016.

        In 2016, Roger Gloor and Patrick  Walther modified the preamp in their
        thesis  \emph{Re-Design eines  integrierten Verst\"arkers  f\"ur einen
        AD-Wandler-IC}, while Marcel Baier and Kevin Niffenegger worked on the
        ADC in their thesis \emph{Sensor-Chip mit Sub-SAR-ADC}.

        These two projects  yielded one redesigned chip  each.  Originally, it
        was intended for these two chips to be more closely examined and their
        performance assessed. However,  due to  an unexpectedly  late delivery
        (end of December 2016), the chip by Burgherr will instead be the focus
        in this  project, while  the two  new chips will  be assessed  in next
        semester's project.
    \item[Problem:]
        Neither  Baier  and Nyffenegger  nor  Gloor  and Walther  performed  a
        comprehensive assessment  of Burgherr's chip, instead  focusing on its
        aspects which were most relevant to their respective projects. We plan
        to rectify this  by performing a comprehensive analysis  of the chip's
        characteristics.
    \item[Personal Interest:]
        Microchip  design  is a  rather  exotic  field, and  opportunities  to
        actually understand and  work on a microchip  are therefore rare. Both
        of us  have been interested  in electronics  and computers for  a long
        time, and  feel compelled to  use this  opportunity to look  under the
        hood, so to speak.
    \item[Own theoretical position:]
        We will work on this project based on the knowledge we have gained and
        will still acquire in the \emph{Analog Circuits} course.
    \item[Question $\ne$ Task:] \texttt{TODO}
    \item[Objective:] \texttt{TODO}
    \item[Methodical Approach:]
        In a first stage, our main  task will be to familiarise ourselves with
        the fundamental  concepts such  as sigma-delta ADCs  and SAR  ADCs, as
        well as the specific implementations  used by the previous teams. They
        have  provided  extensive documentation  for  this. Based  on this,  a
        catalogue  of the  intended measurements  of Burgherr's  chip will  be
        compiled.

        In a second stage, we will perform those measurements and evaluate the
        results.

        In a third stage, based on our conclusions from the previous project's
        reports  as well  as  our  measurements on  Burgherr's  chip, we  will
        prepare for measuring  the two new chips (which  will almost certainly
        need to happen in the next semester).
    \item[Preceding Work:]
        As mentioned, our project is based on the work by other groups.
    \item[Sources:]
        The  primary  resources will  be  the  documentation provided  by  the
        previous  teams, the  book \emph{Analog  Circuit Design}  used in  the
        course \emph{Analog Circuits} as well as any supplementary information
        as needed.
    \item[Scope of Literary Research:]
        Focusing too much on literary research would reduce the amount of time
        available  for  measurements  and  their  evaluation  beyond  what  is
        required. It is therefore  critical to accurately assess  our needs in
        this area, both in content and invested time.

        Certainly, a firm  grasp of $\Sigma\Delta$-ADCs is  needed, along with
        the  other  components  used  in  the  design,  both  in  concept  and
        implementation.

        However, it is difficult to assess  at this stage in the project where
        precisely to  draw the line when  it comes to content. A  limit on the
        time to be  invested will be roughly clear once  we have a preliminary
        timetable (see below).
    \item[Scope of Experiments:] \texttt{TODO}
    \item[Genre and Addressees:]
        The end results  of our work will  be a website, a  presentation and a
        technical report. The report will be  the primary ressource for anyone
        continuing work on this project.

        This  project  is  commissioned  and  supervised  by  Alex  Huber  and
        Hanspeter Schmid  of the  Institute of Microelectronics. They  are its
        main  addressees,  along  with  ourselves and  anyone  else  who  will
        continue this project's work.  All  relevant documents will be written
        in English.
    \item[Preliminary Structuring:] \texttt{TODO}
    \item[Timetable:]
        At this  point, no detailled timetable  has been fixed yet. It  is our
        intention to have a preliminary version by Tuesday, October 11, 2016.
    \item[Risk Assessment:]
        One negative  scenario has  already occurred: The chips  we originally
        intended to  measure and  assess during this  project will  arrive too
        late in the semester for this to be feasible. Therefore, an adjustment
        of this project's scope and objectives has been needed.

        Aside  from this  (and  other  delays beyond  our  influence, such  as
        non-present or faulty equipment), we  believe time mismanagement to be
        the primary danger to our project's success.
\end{description}

%Our main objectives in this project will broadly be as follows:
%
%\begin{itemize}
%    \item
%        Measure and assess the performance of both chips.
%    \item
%        Examine and understand more closely the chip by Baier and Niffenegger,
%        and propose potential improvements for it.
%\end{itemize}
%
%We seek to answer the following questions with this project:
%
%\begin{itemize}
%    \item
%        Have  the design  changes by  Gloor  and Walther  brought the  desired
%        improvements?
%    \item
%        Is  the  design choice  by  Baier  and  Nyffenegger, which  is  highly
%        experimental by their own admission, a worthwile path to pursue?
%\end{itemize}

\end{document}
