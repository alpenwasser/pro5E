\documentclass[a4paper,10pt]{article}


% ---------------------------------------------------------------------------- %
% Packages
% ---------------------------------------------------------------------------- %
\usepackage[T1]{fontenc}
\usepackage[utf8]{inputenc}
%\usepackage[ngerman]{babel}
\usepackage[english]{babel}
\usepackage[a4paper,height=250mm,width=150mm]{geometry}
\usepackage[affil-it]{authblk}
\usepackage{sectsty}
\usepackage{lmodern}
\fontfamily{lmr}\selectfont
\subsectionfont{\fontsize{11}{12}\selectfont}
%\usepackage{xcolor-solarized}
%\pagecolor{solarized-base02}
%\color{solarized-base2}
%\pagecolor{solarized-base2}
%\color{solarized-base02}
%\renewcommand{\familydefault}{\sfdefault}
%\usepackage{siunitx}


% ---------------------------------------------------------------------------- %
% Macros
% ---------------------------------------------------------------------------- %
\def\code#1{\texttt{#1}}


% ---------------------------------------------------------------------------- %
% Title Setup
% ---------------------------------------------------------------------------- %
\title{Project 5 -- Sensor-Microchip}
\author{Raphael Frey, Alexander Murray}
\affil{Institute for Mikroelektronics\\FHNW -- Hochschule f\"ur Technik\\ Studiengang EIT}



% **************************************************************************** %
% Content
% **************************************************************************** %
\begin{document}

% ---------------------------------------------------------------------------- %
\maketitle
% ---------------------------------------------------------------------------- %

% ---------------------------------------------------------------------------- %
% \section{Ziele}
% \label{sec:ziele}
% ---------------------------------------------------------------------------- %

%\vspace{1em}

In his Master Thesis \emph{Sensor Chip}  from  2015, Tobias Burgherr developed a
sensor  integrated  circuit  as  a  base  for  further projects to use. This  IC
consists of a programmable  pre-amplifier  and  a  sigma-delta analog-to-digital
converter with adjustable bandwidth. Typically, sigma-delta ADCs use an internal
1-bit quantizer which gets  boosted  to  the  desired  resolution  of  bits  via
delta-sigma modulation. The goal of his work was to  replace the 1-bit quantizer
with a successive  approximation register (SAR) ADC capable of converting 3 bits
at  a  time  instead  of 1.  This  method  was  referred  to  as  a  ``Multi-bit
quantizer''. A complementary 3-bit DAC was also developed to be used  internally
in the feedback loop of the sigma-delta modulator. Based on his work, two  teams
modified the existing chip in 2016.

In 2016, Roger  Gloor and Patrick Walther modified the  preamp in their thesis
\emph{Re-Design eines  integrierten Verst\"arkers f\"ur  einen AD-Wandler-IC},
while Marcel  Baier and Kevin  Niffenegger worked on  the ADC in  their thesis
\emph{Sensor-Chip mit Sub-SAR-ADC}.

These two projects yielded one redesigned chip each, which are now coming back
from fabrication. Our main objectives in this project will broadly be as follows:

\begin{itemize}
    \item
        Measure and assess the performance of both chips.
    \item
        Examine and understand more closely the chip by Baier and Niffenegger,
        and propose potential improvements for it.
\end{itemize}

Microchip  design is  a rather  exotic  field, and  opportunities to  actually
understand and  work on a microchip  are therefore rare. Both of  us have been
interested in electronics and computers for a long time, and feel compelled to
use this opportunity to look under the hood, so to speak.

We seek to answer the following questions with this project:

\begin{itemize}
    \item
        Have  the design  changes by  Gloor  and Walther  brought the  desired
        improvements?
    \item
        Is  the  design choice  by  Baier  and  Nyffenegger, which  is  highly
        experimental by their own admission, a worthwile path to pursue?
\end{itemize}

Improving the  chip by Baier  and Niffenegger will be  the main focus  of next
semester's  project, which  is why  this  project's focus  is somewhat  biased
towards that chip.

In a  first stage,  our main task  will be to  familiarise ourselves  with the
fundamental concepts such as sigma-delta ADCs and SAR ADCs, as well as the
specific  implementations  used  by  the previous  teams. They  have  provided
extensive documentation  for this. This is particularly  important for Baier's
and  Niffenegger's design,  since  understanding the  system  more closely  is
required  for determining  what to  do  with that  chip  in stage  two of  our
project.

In a  second stage, we will  measure various properties of  the two chips. For
the chip by  Gloor and Walther, this will mean  examining their implementation
of the preamp  stage (gain, speed, bandwidth,  power consumption, \ldots). For
the  chip by  Baier and  Niffenegger, the  relevant characteristics  cannot be
determined with certainty at this point (see above). This will be done once we
have familiarised ourselves with  the system to judge what is  and what is not
relevant  to assess  its  performance, and  what  sorts of  tests  need to  be
performed to gain these insights.

In  a  third  stage, the  chip  by  Baier  and  Niffenegger will  be  examined
more  closely for  ways in  which to  further improve  upon its  design. These
conclusions will be  the starting point for next semester's  project, where we
will  simulate and  design  a new  chip  based on  our  conclusions from  this
semester.

The  primary resources  will be  the  documentation provided  by the  previous
teams, the book  \emph{Analog Circuit Design} used in  the course \emph{Analog
Circuits} as well as any supplementary information as needed.

This project is commissioned and supervised by Alex Huber and Hanspeter Schmid
of the Institute of Microelectronics. They are its main addressees, along with
ourselves and anyone else who will continue this project's work.  All relevant
documents will be written in English.

The primary danger  to this project's success is  time mismanagement. The chip
by Gloor  and Walther will likely  be easier to measure. However,  the chip by
Baier and Niffenegger  will be the foundation for next  semester's project, so
spending too much time on the first chip will have a direct negative impact on
the continuation of this project.  It is therefore crucial to properly plan.

Additionally,  there is  a possibility  of equipment  being faulty  or damaged
(either by somebody else or by us), causing significant delays.

Lastly, it could be that either one or both chips do not function correctly or
even at all (faulty manufacturing, damage by handling, \ldots). Should that be
the case, depending of the severity of  the damage, this might cause delays or
prevent us from carrying out the desired measurements on the chip entirely. To
determine the  course of action,  we would consult  with our advisers  in this
case.

At this point, no detailled timetable  has been fixed yet. It is our intention
to do so by Monday, October 10, 2016.
\end{document}
