% ---------------------------------------------------------------------------- %
\chapter{Test Bench}
\label{chap:testBench}
% ---------------------------------------------------------------------------- %

This  chapter will  present a  brief overview  of the  hardware and  provide a
step-by-step guide to put it into operation.

% ---------------------------------------------------------------------------- %
\section{Hardware and Component List}
\label{sec:hwList}
% ---------------------------------------------------------------------------- %

\todo[inline]{%
        Which devices and components have been used, and for which purposes?%
}

% ---------------------------------------------------------------------------- %
\section{Configuration}
\label{sec:hwList}
% ---------------------------------------------------------------------------- %

\todo[inline]{How was the hardware set up? What was connected to what?}
\todo[inline]{Hardware configuration (buttons, switches, ...}
\todo[inline]{Software configuration (Raspi, python, bash, ...)}

% ---------------------------------------------------------------------------- %
\subsection{Waveform Generators}
\label{subsec:33120A}
% ---------------------------------------------------------------------------- %
\todo[inline]{reference to user manual}
\todo[inline]{scripts, connections}

% ---------------------------------------------------------------------------- %
\subsection{Volt and Amp Meters}
\label{subsec:34465A}
% ---------------------------------------------------------------------------- %

\todo[inline]{reference to user manual}
\todo[inline]{measurement modes, cable connections}

% ---------------------------------------------------------------------------- %
\subsection{DC Power Supply}
\label{subsec:dcPower}
% ---------------------------------------------------------------------------- %
\todo[inline]{reference to user manual}
\todo[inline]{Connections}

% ---------------------------------------------------------------------------- %
\subsection{Oscilloscope}
\label{subsec:oscilloscope}
% ---------------------------------------------------------------------------- %

\todo[inline]{connections}
\todo[inline]{configuration}
\todo[inline]{script}
\todo[inline]{quirks}

% ---------------------------------------------------------------------------- %
\subsection{Raspberry Pi}
\label{subsec:raspi}
% ---------------------------------------------------------------------------- %

\todo[inline]{software installation}
\todo[inline]{measurement programs}
\todo[inline]{pin connections}

% ---------------------------------------------------------------------------- %
\subsection{Sensor Test Board}
\label{subsec:testBoard}
% ---------------------------------------------------------------------------- %

\todo[inline]{DIP Switches}
\todo[inline]{Socket operation}
\todo[inline]{plug connections}
\todo[inline]{safety precautions}

% ---------------------------------------------------------------------------- %
\subsection{Laptop}
\label{subsec:laptop}
% ---------------------------------------------------------------------------- %

\todo[inline]{python scripts}
\todo[inline]{bash scripts}
\todo[inline]{LAN Connection}


%% ---------------------------------------------------------------------------- %
%\section{Hardware Overview}
%\label{sec:hwOverview}
%% ---------------------------------------------------------------------------- %
%
%The   hardware   consists   of   two  primary   components: The   test   board
%(\fref{fig:pcbOverview})and a \raspi~ (\fref{fig:raspi}).
%
%The test board's purpose is to provide all the necessary circuitry for driving
%the sensor  IC in a  convenient package. On its  left-hand side, it  has seven
%industrial plugs  for connecting to  power supplies,  an input signal,  a test
%output for the preamp as well as ground.
%
%At the board's top edge, connections for the externally provided clock and the
%output signal line can be found.
%
%\begin{figure}
%    \centering
%    %\includegraphics[width=\textwidth]{images/pcb/pcbOverview.jpeg}
%    \newcommand{\plugtext}[1]{\textbf{\texttt{\Large{#1}}}}

\begin{tikzpicture}
    \begin{scope}[x={(0mm,135mm)},y={(0mm,79mm)},line width=1pt,cap=round]
        \node[anchor=south west,inner sep=0mm] at (0mm,0mm) {\includegraphics[width=135mm]{images/pcb/pcbOverview.jpeg}};
    \end{scope}
    %\node[fill=black,text=white,anchor=south west] at (0mm,0mm) {\textsf{\Large{VDDD}}};

    \node[%
        %fill=black,
        %fill opacity=0.5,
        text=white,
        text opacity=1,
        rounded corners=1mm,
        anchor=south west] at (4mm,66mm) {\plugtext{VDDD}};

    \node[%
        %fill=black,
        %fill opacity=0.5,
        text=white,
        text opacity=1,
        rounded corners=1mm,
        anchor=south west] at (4mm,56.5mm) {\plugtext{VDDA}};

    \node[%
        %fill=black,
        %fill opacity=0.5,
        text=white,
        text opacity=1,
        rounded corners=1mm,
        anchor=south west] at (4mm,47.5mm) {\plugtext{VGNDA}};

    \node[%
        %fill=black,
        %fill opacity=0.5,
        text=white,
        text opacity=1,
        rounded corners=1mm,
        anchor=south west] at (1mm,37.5mm) {\textbf{\textsf{\large{SIGNAL IN}}}};

    \node[%
        %fill=black,
        %fill opacity=0.5,
        text=white,
        text opacity=1,
        rounded corners=1mm,
        anchor=south west] at (1mm,27.5mm) {\textbf{\textsf{\large{TEST OUT}}}};

    \node[%
        %fill=black,
        %fill opacity=0.5,
        text=white,
        text opacity=1,
        rounded corners=1mm,
        anchor=south west] at (4mm,17mm) {\plugtext{IBIAS}};

    \node[%
        %fill=black,
        %fill opacity=0.5,
        text=white,
        text opacity=1,
        rounded corners=1mm,
        anchor=south west] at (4mm,8mm) {\plugtext{VSS}};
\end{tikzpicture}

%    \caption{Test board overview with its most commonly used connections labeled}
%    \label{fig:pcbOverview}
%\end{figure}
%
%\begin{figure}
%    \centering
%    \missingfigure{Raspberry Pi}
%    \caption{The Raspberry Pi used in this setup}
%    \label{fig:raspi}
%\end{figure}
%
%The chip's top  level pins are listed in  \tref{tab:inputPlugs}. Some of these
%can  be configured  via an  array of  DIP switches,  located between  the chip
%socket  and  the  \signal{CLK}  and  \signal{SIGNAL OUT}  plugs  on  the  test
%board. The most commonly  used among these is the preamp's  gain, which can be
%set to values between \num{-16}  and \num{+16}. The preamp's sign (positive or
%negative) is  a dedicated  switch; its  gain's absolute  values are  listed in
%\tref{tab:dipGain}.
%
%\begin{table}
%    \centering
%    \caption{Sensor chip toplevel pins}
%    \label{tab:inputPlugs}
%    \scriptsize
%    \sisetup{list-pair-separator = { or }}
%    \rowcolors{2}{solarized-base3}{white}
%    \begin{tabular}{>{\fontfamily{jkptt}\selectfont}l>{\fontfamily{jkptt}\selectfont}lp{30mm}lp{30mm}} \\
%    %\begin{tabular}{>{\fontfamily{lmtt}\selectfont}l>{\fontfamily{lmtt}\selectfont}lp{30mm}lp{30mm}} \\
%        \toprule
%        \textnormal{\textsc{Pin \#}} & \textnormal{\textsc{Name}}                  & \textsc{Description} & \textsc{Value} & \textsc{Note} \\
%        \midrule
%        39 & vin                     & analog input signal                         & \SIrange{0.5}{2.5}{\volt} & \\
%        25 & bit\_stream             & digital output signal                       & \SIlist{0;3}{\volt}       & \\
%        34 & clk                     & clock                                       & \SIlist{0;3}{\volt}       & \\
%        36 & sdm\_rst                & pulse to reset the $\Sigma\Delta M$         & \SIlist{0;3}{\volt}       & short pulse is enough (min $T_C$) \\
%        41 & vgnda                   & analog ground                               & \SI{1.5}{\volt}           & \\
%        48 & vrefh                   & high reference voltage                      & \SI{3}{\volt}             & \\
%        47 & vrefl                   & low reference voltage                       & \SI{0}{\volt}             & \\
%        46 & ibias                   & bias current                                & \SI{120}{\micro\ampere}   & $24 \cdot 5 \cdot 1$, internally reduced by 120 \\
%        43 & sc\_amp\_vout\_test\_en & enables test output after preamp            & \SIlist{0;3}{\volt}       & \SI{0}{\volt}: off, \SI{3}{\volt}: on\\
%        44 & sc\_amp\_vout\_test     & test output after preamp                    & \SIrange{0.5}{2.5}{\volt} \\
%        33 & sc\_amp\_rst\_ext\_en   & enables external reset input for the preamp & \SIlist{0;3}{\volt}       & \SI{0}{\volt}: off, \SI{3}{\volt}: on\\
%        32 & sc\_amp\_rst\_ext       & external reset input for the preamp         & \SIlist{0;3}{\volt}       & Internally synced to \signal{clk}. Needs \signal{en} signal. \\
%        35 & sc\_amp\_pos\_neg\_amp  & positive or negative gain                   & \SIlist{0;3}{\volt}       & \SI{0}{\volt}: positive, \SI{3}{\volt}: negative \\
%        28 - 31 & sc\_amp\_csel<3:0> & set gain                                    & \SIlist{0;3}{\volt}       & See table TODO \\
%        27 & sc\_amp\_en             & enable preamp                               & \SIlist{0;3}{\volt}       & \SI{0}{\volt}: off, \SI{3}{\volt}: on\\
%        26 & sah\_sdm\_en            & enable $\Sigma\Delta M$ S/H bock            & \SIlist{0;3}{\volt}       & \SI{0}{\volt}: off, \SI{3}{\volt}: on\\
%        42 & vdda                    & analog positive power supply                & \SI{3}{\volt}             & \\
%        37 & vddd                    & digital positive power supply               & \SI{3}{\volt}             & \\
%        40 & vss                     & analog and digital negative power supply    & \SI{0}{\volt}             & \\
%        \bottomrule
%    \end{tabular}
%    \sisetup{list-pair-separator = { and }}
%\end{table}
%
%\begin{table}
%    \centering
%    \caption{DIP switch settings for amplifier gain}
%    \label{tab:dipGain}
%    \scriptsize
%    \begin{tabular}{%
%            >{\fontfamily{jkptt}\selectfont}r
%            >{\fontfamily{jkptt}\selectfont}r
%            >{\fontfamily{jkptt}\selectfont}r
%            >{\fontfamily{jkptt}\selectfont}r|
%            >{\fontfamily{jkptt}\selectfont}r}
%        \toprule
%        \signal{sc\_amp\_en(3)} &
%        \signal{sc\_amp\_en(2)} &
%        \signal{sc\_amp\_en(1)} &
%        \signal{sc\_amp\_en(0)} &
%        Gain \\
%        \midrule
%        1 & 1 & 1 & 1 &  1 \\
%        0 & 1 & 1 & 1 &  2 \\
%        0 & 0 & 1 & 1 &  4 \\
%        0 & 0 & 1 & 1 &  8 \\
%        0 & 0 & 0 & 1 & 16 \\
%        \bottomrule
%    \end{tabular}
%\end{table}
%
%
%% ---------------------------------------------------------------------------- %
%\clearpage
%\section{Putting the Sensor IC Into Operation}
%\label{sec:ICintoOperation}
%% ---------------------------------------------------------------------------- %
%
%This  section  will  detail  the  steps  needed to  put  the  Sensor  IC  into
%operation. Some  of  this  information  will be  redundant  with  the  reports
%\cite{ref:burgherr},  \cite{ref:gloor} and  \cite{ref:baier},  but  we aim  to
%provide a  convenient guide for this  process in a single  place, thus sparing
%future teams the effort of needing  to assemble this information from multiple
%sources, which is both a time-consuming and error-prone process.
%\todo[noline]{Make direct references to pages/tables in other reports?}
%\todo[noline]{Safety precations}
%
%Fundamentally, putting the chip into operation consists of the following steps:
%\begin{itemize}\tightlist
%    \item
%        Set up \raspi.
%    \item
%        Determine bias resistor setting.
%    \item
%        Set   DIP   switches   to   the  appropriate   values   according   to
%        \tref{tab:dipGain}.
%    \item
%        Connect supply voltages and bias current supply.
%    \item
%        Connect \raspi~and test board.
%\end{itemize}
%
%% ---------------------------------------------------------------------------- %
%\subsection{Setting up the Raspberry Pi}
%\label{subsec:raspiInstall}
%% ---------------------------------------------------------------------------- %
%
%\todo[inline]{%
%    Give a more detailled guide on what to install on the \raspi~ and how than
%    in the previous reports
%}
%
%\todo[inline]{RaspiPower supply}
%\todo[inline]{External connections for Raspi}
%\todo[inline]{%
%    Controlling the  function generator  for \signal{SIGNAL  IN} line  via the
%    \raspi.
%}
%
%% ---------------------------------------------------------------------------- %
%\subsection{Hardware Setup}
%\label{subsec:hardwareSetup}
%% ---------------------------------------------------------------------------- %
%
%Operating the test board requires the following components:
%
%\begin{itemize}\tightlist
%    \item
%        direct voltage source: \SI{3}{\volt}
%    \item
%        direct voltage source: \SI{1.5}{\volt}
%    \item
%        Direct  current  source: \SI{120}{\micro\ampere}.  \emph{Note:} If  no
%        direct current source  is available, as in our case,  a direct voltage
%        source in combination with  a multimeter for controlling/measuring the
%        current can be used.
%    \item
%        function generator for \signal{clk}
%    \item
%        Direct  and  alternating  voltage  source  for  input  signal  voltage
%        (between \SI{0.5}{\volt} and \SI{2.5}{\volt}).
%    \item
%        An oscilloscope for monitoring the preamp's output on the \signal{TEST
%        OUT} pin, as well as the raw bit stream.
%\end{itemize}
%
%The complete setup is depicted in \fref{fig:experimentDiagram}.
%
%\begin{figure}
%    \sisetup{range-phrase = { \ldots }}
\sisetup{list-pair-separator = {/}}

\begin{circuitikz}[x=1mm,y=1mm]

    % ------------------------------------------------------------------------ %
    % Test Board
    % ------------------------------------------------------------------------ %
    \draw (0,0) -- (50,0) -- (50,29) -- (0,29) -- cycle;

    % Left-hand side Plugs
    \foreach \y in {0,...,6} {%
        \draw (0,\y*3.5+2.5) -- (8,\y*3.5+2.5) -- (8,\y*3.5+5.5) -- (0,\y*3.5+5.5) -- cycle;
    };

    % CLK plug
    \draw (23,27.5) -- (27,27.5) -- (27,28.5) -- (23,28.5) -- cycle;
    \foreach \x in {0,...,3} {%
        \fill (23.25+\x,27.75) -- (23.25+\x+0.5,27.75) -- (23.25+\x+0.5,28.25) -- (23.25+\x,28.25) -- cycle;
    }

    % Signal out Plug
    \draw (29,27.5) -- (33,27.5) -- (33,28.5) -- (29,28.5) -- cycle;
    \foreach \x in {0,...,3} {%
        \fill (29.25+\x,27.75) -- (29.25+\x+0.5,27.75) -- (29.25+\x+0.5,28.25) -- (29.25+\x,28.25) -- cycle;
    }

    % DIP switch housing
    \draw (22,24) -- (32,24) -- (32,26) -- (22,26) -- cycle;
    \foreach \x in {1,...,9} {%
        \draw (\x+22,24) -- (\x+22,26);
    }

    % DIP Switches
    \foreach \x in {0,...,9} {%
        \fill (22+\x+0.25,24.5) -- (22+\x+0.75, 24.5) -- (22+\x+0.75,25) -- (22+\x+0.25,25) -- cycle;
    }


    % ------------------------------------------------------------------------ %
    % RasPi
    % ------------------------------------------------------------------------ %
    \draw (10,50) -- (50,50) -- (50,70) -- (10,70) -- cycle;

    % GPIO
    \foreach \x in {0,...,39} {%
        \fill (49.5-\x*0.6667-0.25,50.25) -- (49.5-\x*0.66675-0.25-0.35,50.25) -- (49.5-\x*0.66675-0.25-0.35,50.5) -- (49.5-\x*0.6667-0.25,50.5);
        \fill (49.5-\x*0.6667-0.25,50.875) -- (49.5-\x*0.66675-0.25-0.35,50.875) -- (49.5-\x*0.66675-0.25-0.35,51.125)   -- (49.5-\x*0.6667-0.25,51.125);
    }

    % ------------------------------------------------------------------------ %
    % Voltage and Current Sources
    % ------------------------------------------------------------------------ %

    % Supply Voltages
    \draw (0,25) -- (-5,25) node[circ] {}-- (-20,25) to[american voltage source,l_=\scriptsize{\SI{3}{\volt}},-*] (-20,40);

    \draw (-5,25) -- (-5,21.5) -- (0,21.5);
    \draw (0,18) -- (-40,18) to[american voltage source,l_=\scriptsize{\SI{1.5}{\volt}},-*] (-40,40);

    \draw (-60,40) to[american current source,l^=\scriptsize{\SI{120}{\micro\ampere}},*-] (-60,11) -- (0,11);

    % GND Line
    \draw (0,4) -- (-70,4) -- (-70,40) -- (25.125,40) -- (25.125,50.25);

    % CLK
    \draw (0,40) node[circ]{} -- (0,35) to[square voltage source,-*] (23.5,35) -- (23.5,28);
    \draw (23.5,35) -- (28.375,35) -- (28.375,50.25);

    % ------------------------------------------------------------------------ %
    % Signal out line
    % ------------------------------------------------------------------------ %
    \draw (30.5,28) -- (30.5,50.25);


    \draw (-70,4) node[ground]{} node[circ]{};

    %\foreach \ini [evaluate=\ini as \inieval using 2*\ini] in {0,...,6}
    %\draw[ultra thick,cyan] (\inieval,0) -- ++(0,1) -| (\inieval+1,0) -- (\inieval+2,0);
\end{circuitikz}


%    \caption{%
%        Experimental setup  with connections.  \emph{Note}: Drawing not
%        finalized yet.%
%    }
%    \label{fig:experimentDiagram}
%\end{figure}
%
%The  voltage  on  the  bias  input  must be  set  so  that  the  bias  current
%is  \SI{120}{\micro\ampere}. The  current  can  be adjusted  with  a  variable
%resistor. \cite{ref:burgherr} outlines how the  value for this resistor can be
%calculated. To ensure correct  operation, it is highly  recommended to monitor
%the current on the bias input pin  with a current meter and adjust the voltage
%source as needed to  achieve a current of \SI{120}{\micro\ampere}.\todo{actual
%R/V values used in our experiments}
%% Burgherr: I-32: Bias current
%
%Additionally, we strongly advise validating the  DC input voltages with a volt
%meter and not blindly trusting the given voltage source's indicator.
%
%The final setup, as used in the experiments for this report, is depicted in \fref{fig:completeSetup}.
%
%\begin{figure}
%    \missingfigure{Complete setup as used in the experiments for this report (photograph, not diagram)}
%    \caption{the setup as used in the experiments for this report}
%    \label{fig:completeSetup}
%\end{figure}
%
%An \emph{Aim TTi MX100TP} laboratory  DC power supply (\fref{fig:dcSupply}) is
%used  as input  for the  \SI{3}{\volt}, \SI{1.5}{\volt}  and the  bias current
%lines.
%
%\begin{figure}
%    \missingfigure{DC Power Supply}
%    \caption{DC Power Supply}
%    \label{fig:dcSupply}
%\end{figure}
%
%The  bias  current  is  monitored with  an  \emph{Agilent  U1253B}  multimeter
%(\fref{fig:agilentMultimeter}),  while   a  \emph{Keysight   34465A}  tabletop
%(\fref{fig:keysightMultimeter})  multimeter is  used for  measuring the  input
%signal voltage (in order not stay within the sensible range of \SI{0.5}{\volt}
%and \SI{2.5}{\volt},  below and  above which  the ADC  enters saturation). The
%output  bitstream   is  monitored   with  a  \emph{LeCroy   waveRunner  6100A}
%oscilloscope.
%
%\begin{figure}
%    \missingfigure{DC Power Supply}
%    \caption{DC Power Supply}
%    \label{fig:agilentMultimeter}
%\end{figure}
%
%\begin{figure}
%    \missingfigure{Keysight Multimeter}
%    \caption{Keysight tabletop multimeter}
%    \label{fig:keysightMultimeter}
%\end{figure}
%
%\begin{figure}
%    \missingfigure{LeCroy Oscilloscope}
%    \caption{LeCroy Oscilloscope}
%    \label{fig:lecroyOscilloscope}
%\end{figure}
%
%Two    \emph{Hewlett   Packard    33120A}   arbitrary    waveform   generators
%(\fref{fig:HPwave}) are used  to provide the system's clock  and input signal,
%respectively. The device  used for generating  the input signal  is controlled
%remotely  via the  \raspi\todo{Document  Python scripts  used for  this}. This
%reduces the  time required  for performing  experiments significantly. Further
%efficiency gains  could be made by  replacing the DIP switches  with something
%which can be controlled remotely (the only thing requiring manual intervention
%at  that point  would  be the  replacing  of the  chip  itself when  measuring
%multiple samples).
%
%\begin{figure}
%    \missingfigure{Hewlett Packard Arbitrary Waveform Generator}
%    \caption{Hewlett Packard Arbitrary Waveform Generator}
%    \label{fig:HPwave}
%\end{figure}
%
%Appendix \ref{sec:HPwave}  details the configuration process  for the function
%generator, starting on page \pageref{sec:HPwave}.
