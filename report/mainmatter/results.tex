% ---------------------------------------------------------------------------- %
\chapter{Results}
\label{chap:results}
% ---------------------------------------------------------------------------- %


\begin{itemize}\tightlist
    \item
        Saturation: Lower boundary: bitstream without any input signal
    \item
        Saturation: Upper boundary: Find meaningful equivalent to "no input" at lower boundary
    %\item
    %    Why do we perform DC measurements? (linearity, offset)
    %\item
    %    Why do we  perform AC measurements w/ harmonics? (What  sort of system
    %    can be measured with this chip, how fast is the plant?)
    %\item
    %    Why do we perform SNR measurements?
    \item
        What are potential  measurements which might be of  interest but which
        we  could  not perform  (and  the  reasons  for  which they  were  not
        performed)?
    \item
        What would need to be done to be able to perform those measurements?
    %\item
    %    What kind of measurements were \emph{not} performed, and why?
    %\item
    %    Which measurements were performed?
    %\item
    %    Why are these measurements relevant?
    %\item
    %    What kind of results are we expecting? (Simulation results)
    %\item
    %    What are our actual results?
    %\item
    %    What are potential causes for the differences? (Potentially split this off
    %    into a separate chapter titled \emph{Analysis} or something similar.)
\end{itemize}

%\todo[inline]{Statistical analysis of results}

% --------------------------------------------------------------------------- %
\section{Pre-Amplifier: DC Measurements}
\label{sec:preAmpDC}
% --------------------------------------------------------------------------- %

The preamplifier was directly  measured, isolated from the \sdm, by applying a
number of constant voltages on the input and recording the output voltage with
an oscilloscope on the \signal{TEST OUT} pin.

Since the  preamp  uses  a  switching  capacitor  implementation,  the  output
waveform is of  course  a  square  wave  of  sorts,  as  can be seen in figure
\todo[inline]{generate}.

A first observation to make are the slow rise and fall-times. Indeed,  this is
one  of the major limiting factors of the chip's performance;  If  the  target
voltage can't be reached within half of the clock period (before it  is  reset
again),  the  \sdm will end up  converting  a  lower  voltage  than  expected.

For  every  chip,  for  every sampling frequency, for every gain, inverted and
non-inverted,  11  different  input voltages were applied and the  output  was
measured.  All  of  the  data  was  then fitted using a simple first order  RC
low-pass charge/discharge model, described by the following python code:

\begin{minted}{python}
def preamp_curve(t, amp, amp_offset, period, t_offset, duty_cycle, tau1, tau2):
    def discharge_segment(t, amp, tau):
        return amp * np.exp(-t / tau)
    def charge_segment(t, amp, tau):
        return amp * (1.0 - np.exp(-t / tau))
    t_local = (t+t_offset) % period
    t_switch = period * duty_cycle
    def preamp_curve_single():
        for t_value in t_local:
            if t_value < t_switch:
                yield charge_segment(t_value, amp, tau1) + amp_offset
            else:
                yield discharge_segment(t_value - t_switch, amp, tau2) + amp_offset
    return np.array([x for x in preamp_curve_single()])
\end{minted}

The initial parameters for the fit were obtained by first smoothing  the  data
(using a  Savitzky-Golay  filter)  and  finding  the  transition  points. This
process is illustrated  in  figure \todo[inline]{generate}. This was necessary
because there are a lot  of  local minima the fit could fall into. The initial
period, duty cycle, and phase shift can be extracted from the distance between
the transition lines. The initial amplitude and  offset can be determined with
a min/max search.

The  most  interesting  parameters  yielded from  the  fit  are  $\Tau_1$  and
$\Tau_2$, which give us a rough idea of the transconductance $g_m$ of the OTA.
The gain can also be easily extracted after the fit by  performing  a  min/max
search on the  fitted  curve. This is illustrated in figure \todo[inline]{generate} as
the horizontal line.

It can  be  observed  that the transconductance $g_m$ is largest when near the
reference  voltage  $V_{ref}=\SI{1.5}{\volt}$  and decreases with higher input
amplitudes. This effect can  be seen in figure \todo[inline]{generate a figure with 11
overlapping preamp signals}. Furthermore, in the same figure,  one can observe
slewing on the larger signals. This effect could be  included in the fit model
above in future evaluations.

By combining the individual  $\Tau$ constants and plotting them in function of
the  input voltage, we see the effect of $g_m$ decreasing more clearly (figure
\todo[inline]{generate  tau  figures}).  Here,   the   results   from  10  chips  were
statistically averaged to get a better result.

\todo[inline]{is it frequency dependent? Shouldn't be}

By  plotting  the output voltage from the preamp  in  function  of  the  input
voltage and fitting a linear function $y=mx+q$ to the data, we expect a linear
function who's  incline is equal to that of the configured gain. Additionally,
there we expect a small offset to exist, due to the imperfections  of reality.
This  process  is   visualised   in   figure   \todo[inline]{generate   this  figure}.

Figure \todo[inline]{generate this} shows all positive gains (1,  2,  4,  8,  16)  for
every chip (x axis) in function of $fs$.

% --------------------------------------------------------------------------- %
\section{Sigma-Delta Converter: DC Measurements}
\label{sec:sigdelDC}
% --------------------------------------------------------------------------- %

% --------------------------------------------------------------------------- %
\section{Complete System: DC Measurements}
\label{sec:systemDC}
% --------------------------------------------------------------------------- %

% --------------------------------------------------------------------------- %
\section{Pre-Amplifier: AC Measurements}
\label{sec:preAmpAC}
% --------------------------------------------------------------------------- %

% --------------------------------------------------------------------------- %
\section{Sigma-Delta Converter: AC Measurements}
\label{sec:sigdelAC}
% --------------------------------------------------------------------------- %

% --------------------------------------------------------------------------- %
\section{Complete System: AC Measurements}
\label{sec:systemAC}
% --------------------------------------------------------------------------- %

%\todo[inline]{Data from Oscilloscope}

%\begin{table}
%    \centering
%    \fontfamily{jkpss}\selectfont
%    \caption{DC Measurements}
%    \label{tab:dcMeas}
%    \begin{tabular}{ccc}
%        \toprule
%        {$V_{\textnormal{in}}$ (\si{\volt})} &
%        {$G_{\textnormal{Preamp}}$}          &
%        {$f_{\textnormal{s}}$ (\si{\kilo\hertz})}         \\
%        \midrule
%        0.5 & {1, 2, 4, 6, 8, 16} & {32, 256} \\
%        0.7 & {1, 2, 4, 6, 8, 16} & {32, 256} \\
%        0.9 & {1, 2, 4, 6, 8, 16} & {32, 256} \\
%        1.1 & {1, 2, 4, 6, 8, 16} & {32, 256} \\
%        1.3 & {1, 2, 4, 6, 8, 16} & {32, 256} \\
%        1.5 & {1, 2, 4, 6, 8, 16} & {32, 256} \\
%        1.7 & {1, 2, 4, 6, 8, 16} & {32, 256} \\
%        1.9 & {1, 2, 4, 6, 8, 16} & {32, 256} \\
%        2.1 & {1, 2, 4, 6, 8, 16} & {32, 256} \\
%        2.3 & {1, 2, 4, 6, 8, 16} & {32, 256} \\
%        2.5 & {1, 2, 4, 6, 8, 16} & {32, 256} \\
%        \bottomrule
%    \end{tabular}
%\end{table}
