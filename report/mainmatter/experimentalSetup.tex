% ---------------------------------------------------------------------------- %
\chapter{Experimental Setup}
\label{ch:experimentalSetup}
% ---------------------------------------------------------------------------- %

\begin{itemize}\tightlist
    \item
        How does the device itself look, physically?
    \item
        What considerations must  be taken when trying to put  the device into
        operation?
    \item
        How does the resulting setup look, w/ all devices and their settings?
    \item
        How were the measurements conducted (methodology)?
    \item
        How were the results obtained and evaluated?
\end{itemize}

\todo[inline]{Fundamental experimental setup w/ Raspbi, test board, PC}
\missingfigure{Block Diagram: Raspbi, Test Board, PC}


% ---------------------------------------------------------------------------- %
\section{Putting the Sensor IC Into Operation}
\label{sec:ICintoOperation}
% ---------------------------------------------------------------------------- %

This  section  will  detail  the  steps  needed to  put  the  Sensor  IC  into
operation. Some  of  this  information  will be  redundant  with  the  reports
\cite{ref:burgherr},  \cite{ref:gloor} and  \cite{ref:baier},  but  we aim  to
provide a  convenient guide to  this process in  a single place,  thus sparing
future teams the effort of needing  to assemble this information from multiple
sources.
\todo[noline]{Make direct references to pages/tables in other reports?}
\todo[noline]{handling with care and all that stuff}


% ---------------------------------------------------------------------------- %
\subsection{Basic Layout of PCB}
\label{subsec:PCBLayout}
% ---------------------------------------------------------------------------- %

A test board, depicted in figure  \todo{test board fig ref} has been developed
in  \todo{Insert  Reference  to  p.1-32  in  Burgherr}  to  enable  convenient
examination of the sensor IC.
\todo[noline]{How to operate chip socket}

\missingfigure{Overview of PCB, sections referenced}
\missingfigure{PCB Schematic}
\missingfigure{Chip Socket}


On the left-hand side, there are seven industrial plugs\todo{correct name?}


% ---------------------------------------------------------------------------- %
\subsection{Supply Voltages, Input Signal and PCB Settings}
\label{subsec:supplyVoltagesAndSettings}
% ---------------------------------------------------------------------------- %

\todo[inline]{bias current resistors}
\todo[inline]{supply voltages, Power Supplies}
\todo[inline]{DIP Switches}
\todo[inline]{Clock: between 0V and 3V, not -1.5 and +1.5}
\todo[inline]{Output impedance of signal generators}
\todo[inline]{Jumpers for some sockets}
\missingfigure{Bias Current Resistors}
\missingfigure{Power Supplies}
\missingfigure{Signal Generators}
\missingfigure{Cabling}
\missingfigure{Multimeters}
\missingfigure{Oscilloscope}


% ---------------------------------------------------------------------------- %
\subsection{Setting up the Raspberry Pi}
\label{subsec:setupRaspbi}
% ---------------------------------------------------------------------------- %

\todo[inline]{installation}
\todo[inline]{voltages, connections w/ test board}


% ---------------------------------------------------------------------------- %
\subsection{Performing Measurements}
\label{subsec:performingMeasurements}
% ---------------------------------------------------------------------------- %

\todo[inline]{Reading Data on Raspbi}
