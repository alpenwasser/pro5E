% ---------------------------------------------------------------------------- %
\chapter{Conclusions and Outlook}
\label{chap:conclusions}
% ---------------------------------------------------------------------------- %

This project's  main objectives  were to create  and document  a comprehensive
test bench  and the corresponding  data processing  tools, and to  measure and
assess  the performance  and  characteristics of  the  \sdm.  Originally,  the
plan  was  to  to  measure  new  chip  designs  which  build  upon  the  first
iteration. However, as  we found out during  the project, the new  chips would
not arrive in  time to perform these measurements. Therefore, we  fell back to
the original chip design in our measurements.

Developing the test  bench  and  the  data processing tools turned out to be a
noticeably more  complex undertaking than originally anticipated, and progress
was not as swift as we had hoped.  Additionally,  some time was lost due to an
incorrect component having been soldered onto  the  test board, requiring time
for finding and fixing the error. Lastly, we  occasionally  got hung up on the
wrong  details,  investing  time  without  generating   much   usable  output.

This resulted  in  some  delays,  which  means  not  all  objectives have been
achieved  at  this point. Specifically, mapping the measurement results to the
internal  architecture  of  the  chip  via simulations still needs to be done.
However, the test bench and data processing tools  have  been  developed  to a
satisfying and  very  usable  state,  allowing  for  quick  measurement  of  a
multitude of chips and analysis of the resulting data.

As  was  already established in the previous  work\cite{ref:gloor},  the  poor
characteristics  of  the pre-amplifier  are  the  main  cause  for  diminished
performance  of  the  device.  In  this  work,  measurements  have  shown  the
pre-amplifier having a  rather  large  offset  and a gain that varies strongly
depending  on  the operating frequency. Deviations in effective  gain  can  be
traced back to slow rise and  fall  times  of the switched capacitor OTA; When
the operating  frequency  is  close to the limit (which turns out to be around
\SI{256}{\kilo\hertz}), the  target output voltage of the pre-amplifier can no
longer  be  reached within one clock cycle and the  converted  value  deviates
strongly as a result.

There  are  multiple possible reasons for the slow rise and  fall  times.  The
transconductance of the OTA might  be  too  low. Increasing it leads to better
performance,  but  at  the  cost  of  higher power consumption. Decreasing the
values of the switched capacitors can lead to faster rise and fall times,  but
at the cost of higher noise and more susceptibility to parasitic capacitances.
Considering   the   fact   that   the   switched   capacitors   are   in   the
\SI{100}{\femto\farad}-range, and considering  the fact that the output of the
pre-amplifier is directly connected  to a lead on the chip (which is typically
in  the  \SI{}{\pico\farad}-range),  the  lead  may  very  well  have a strong
negative influence  on  the  pre-amplifier's  performance.  This  hasn't  been
tested, but may be worth considering.

Without the aid of simulations, we can't make more decisive conclusions.

On the bright side, though, the \sdm~appears to perform quite well.

- objective: develop test bench and data processing stuff
- success: moderately: test bench ok, scripts okay-ish, no simulations
- why not? time delays
- how to proceed into p6?


\begin{itemize}\tightlist
    \item
        Short recap: What did we want to achieve?
    \item
        Were we successful?
    \item
        If not: Why? How do we proceed from here?
    \item
        How to proceed from here into P6?
\end{itemize}
