% ---------------------------------------------------------------------------- %
\chapter{Conclusions and Outlook}
\label{chap:conclusions}
% ---------------------------------------------------------------------------- %

This project's  main objectives  were to create  and document  a comprehensive
test bench  and the corresponding  data processing  tools, and to  measure and
assess  the performance  and  characteristics of  the  \sdm.  Originally,  the
plan  was  to  to  measure  new  chip  designs  which  build  upon  the  first
iteration. However, as  we found out during  the project, the new  chips would
not arrive in  time to perform these measurements. Therefore, we  fell back to
the original chip design in our measurements.

Developing the  test bench and  the data processing tools  turned out to  be a
noticeably more complex undertaking  than originally anticipated, and progress
was not as swift as we had hoped.   Additionally, some time was lost due to an
incorrect component having  been soldered onto the test  board, requiring time
for finding and fixing the error. Lastly,  due to our lack of familiarity with
the subject, we occasionally got hung  up on the wrong details, investing time
without generating much usable output.

This  resulted in  some  delays,  which means  not  all  objectives have  been
achieved  at  this point. Specifically,  mapping  the  measurement results  to
the  internal architecture  of  the chip  via simulations  still  needs to  be
done.  However, the  test bench and data processing tools  have been developed
to a  satisfying and very  usable state, allowing  for quick measurement  of a
multitude of chips and efficient analysis of the resulting data.

As has  already been  established in the  previous work  \cite{ref:gloor}, the
poor characteristics  of the pre-amplifier  are the main cause  for diminished
performance  of  the  device.   In  this work,  measurements  have  shown  the
pre-amplifier having  a rather large  offset and  a gain that  varies strongly
depending  on the  operating frequency. Deviations  in effective  gain can  be
traced back to  slow rise and fall  times of the switched  capacitor OTA; When
the operating frequency is close  to the limit (around \SI{256}{\kilo\hertz}),
the target output voltage of the pre-amplifier can no longer be reached within
one clock cycle and the converted value deviates strongly as a result.

There are  multiple possible reasons  for the slow  rise and fall  times.  The
transconductance of  the OTA  might be  too low. Increasing  it would  lead to
better performance,  but at the  cost of higher  power consumption. Decreasing
the values of the switched capacitors can  lead to faster rise and fall times,
but  at  the  cost  of  higher noise  and  more  susceptibility  to  parasitic
capacitances.  Considering  the fact that  the switched capacitors are  in the
\SI{100}{\femto\farad}-range, and taking  into account that the  output of the
pre-amplifier is directly connected to a  lead on the chip (which is typically
in  the  \SI{}{\pico\farad}-range), the  lead  may  very  well have  a  strong
negative  influence  on the  pre-amplifier's  performance.   This hasn't  been
tested, but may be worth considering.

Without the  aid of simulations,  we can't  draw more concrete  conclusions at
this point.   On the bright side,  though, the \sdm~itself appears  to perform
quite well.

Going into P6, we currently see the following tasks as critical:

\begin{itemize}\tightlist
    \item
        Measuring and evaluting  the new chip designs,  as originally intended
        for this project. Thanks to the test bench and data processing scripts
        developed  during  this  project,  this should  be  achievable  fairly
        quickly.
    \item
        Performing simulations in order to  gain a deeper understanding of how
        the circuit  actually works, and  gaining insight into  \emph{why} the
        measurement results look as they do.
    \item
        Based on  this, isolate critical areas of the design and  improve upon
        them.
    \item
        Evaluate whether  or not  the \raspi~ is  actually sufficient  for the
        task for which  it is being used,  and if not, what  measures to take.
        This  could mean  either optimizing  the software  or, in  the extreme
        case, replacing the hardware with something of higher performance.
\end{itemize}
