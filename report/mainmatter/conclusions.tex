% ---------------------------------------------------------------------------- %
\chapter{Conclusions and Outlook}
\label{chap:conclusions}
% ---------------------------------------------------------------------------- %

This project's  main objectives  were to create  and document  a comprehensive
test bench  and the corresponding  data processing  tools, and to  measure and
assess  the performance  and  characteristics of  the  \sdm.  Originally,  the
plan  was  to  to  measure  new  chip  designs  which  build  upon  the  first
iteration. However, as  we found out during  the project, the new  chips would
not arrive in  time to perform these measurements. Therefore, we  fell back to
the original chip design in our measurements.

Developing the  test bench and  the data processing tools  turned out to  be a
noticeably more complex undertaking  than originally anticipated, and progress
was not as swift  as we had hoped. Additionally, some time was  lost due to an
incorrect component having  been soldered onto the test  board, requiring time
for finding  and fixing  the error. Lastly,  we occasionally  got lost  in the
wrong detail, investing time without generating much usable output.

This  resulted in  some  delays,  which means  not  all  objectives have  been
achieved  at  this point. Specifically,  mapping  the  measurement results  to
the  internal architecture  of  the chip  via simulations  still  needs to  be
performed. However,  the  test  bench  and data  processing  tools  have  been
developed  to  a  satisfying  and   very  usable  state,  allowing  for  quick
measurement of a multitude of chips and analysis of the resulting data.

%%% ALEX STUFF


- objective: develop test bench and data processing stuff
- success: moderately: test bench ok, scripts okay-ish, no simulations
- why not? time delays
- how to proceed into p6?


\begin{itemize}\tightlist
    \item
        Short recap: What did we want to achieve?
    \item
        Were we successful?
    \item
        If not: Why? How do we proceed from here?
    \item
        How to proceed from here into P6?
\end{itemize}
