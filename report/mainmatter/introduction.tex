% ---------------------------------------------------------------------------- %
\chapter{Introduction}
\label{ch:introduction}
% ---------------------------------------------------------------------------- %

%\begin{itemize}\tightlist
%    \item
%        What has been done so far?
%    \item
%        What are our objectives?
%    \item
%        What are the  theoretical foundations (do not give here,  but refer to
%        literature instead)?
%    \item
%        How does this tie into P6?
%\end{itemize}


In his  master thesis \emph{Sensor Chip}  \cite{ref:burgherr}, Tobias Burgherr
developed a  sensor chip for converting  an analog input voltage  to a digital
output (bitstream). Broadly speaking, the chip consists of a pre-amplifier and
an analog-to-digital converter.

Based on this work, two teams then each worked to improve the existing design;
one  focusing on  the  preamp  \cite{ref:gloor}, the  other  team  on the  ADC
\cite{ref:baier}.

Our primary  objective in this  project will be to  develop a test  suite with
which to  conveniently and accurately measure  the first chip, as  well as the
chips  developed by  the  two successor  teams. While  some measurements  have
already been performed on the first chip,  the amount of tests run was limited
due to time  constraints by the respective  teams.  We aim to  expand on those
measurements with  out test suite,  as well as use  them as a  reference point
against which we can compare our findings.

Once the  two new  chips become  available, the methodology  used in  our test
suite will allow efficient and accurate measuring  of the two new chips in our
next project.

Where sensible, our measurements will be compared against simulations.

This  report  focuses  primarily  on measurements  and  results  analysis. For
this  reason, no  extensive theoretical  chapters are  provided, and  we refer
the  reader  to  the  reports  by  the  preceding  teams  \cite{ref:burgherr},
\cite{ref:gloor} and \cite{ref:baier} as well as textbooks.

Chapter  \ref{chap:experimentalSetup}  presents  an  overview  of  the  system
hardware   and   software,  and   details   how   to   put  the   setup   into
operation. Chapter   \ref{chap:measurements}   contains   a   description   of
measurement  methodology and  rationale,  answering  the questions  ''\emph{What
did   we  measure,   how   and  why?}``. In   Chapter  \ref{chap:results},   the
results  are   presented  and  compared  with   simulations. Finally,  Chapter
\ref{chap:conclusions} presents our conclusions, which will form the basis for
further work in our bachelor thesis.
