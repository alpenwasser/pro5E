% ---------------------------------------------------------------------------- %
\chapter*{Abstract}
\label{ch:abstract}
% ---------------------------------------------------------------------------- %

As an ongoing  project at the Institute of Microelectronics,  a \sdm~ has been
in development over the past few years. Our project's objective was to develop
a comprehensive  test and  data processing suite  to efficiently  and reliably
assess the performance  of various verions of these chips. In  a second stage,
the results of these measurements are to be used to further improve the \sdm's
design.

The test bench allows efficient measuring  of multiple chips in a short amount
of time and quickly evaluating the  results.  The measurement process has been
largely automated with scripts, requiring  little manual intervention.  In our
project, this  has resulted  in roughly \num{4000}  measurements, representing
about  \num{500}  million  measurement   points  and  requiring  approximately
\num{12} gigabytes  of storage space.  The  setup has been documented  so that
future groups can rebuild it and reproduce our results.

For the chip design which has been  evaluated, the slow rise and fall times of
the  pre-amplifier  are  the  limiting  factor  in  performance. The  sampling
frequency  limit is  around \SI{256}{\kilo\hertz},  which is  where the  OTA's
gain  starts decreasing  drastically.  Possible  solutions include  increasing
the  transconductance, resulting  in  a higher  power consumption,  decreasing
the  switched capacitor  values, which  increases susceptibility  to parasitic
factors,  or decreasing  the  output  load, perhaps  by  buffering the  output
instead of connecting it directly to one of the chip's leads.
