% ---------------------------------------------------------------------------- %
\chapter*{Abstract}
\label{ch:abstract}
% ---------------------------------------------------------------------------- %

As an ongoing  project at the Institute of Microelectronics,  a \sdm~ has been
in development over the past few years. This project's objective was to develop
a comprehensive  test and  data processing suite  to efficiently  and reliably
assess the performance  of various verions of these chips. In  a second stage,
the results of these measurements are to be used to further improve the \sdm's
design.

We  used the testbench  to  measure  ten  chips  in  a  few  days  in  various
configurations. The  measurement  process  has  been  largely  automated  with
scripts,  requiring  little manual intervention.  In  our  project,  this  has
resulted   in   roughly   \num{4000}    measurements,    representing    about
\num{500000000}   measurement  points  and  requiring  approximately  \num{12}
gigabytes  of  storage  space.  The setup has been documented so  that  future
groups can rebuild it and reproduce our results.

The slow rise and fall times  of  the pre-amplifier are the limiting factor in
this  chip's  performance. The sampling frequency limit turns out to be around
\SI{256}{\kilo\hertz}.  This  is   where  the  OTA's  gain  starts  decreasing
drastically.  Possible  solutions  include  increasing  the  transconductance,
resulting in  a  higher  power  consumption, decreasing the switched capacitor
values, which increases susceptibility to parasitic factors, or decreasing the
output load, perhaps by buffering the output instead of connecting it directly
to one of the chip's leads.


